% -*- coding:utf-8 -*-
\documentclass[12pt]{article}
\usepackage[utf8]{inputenc}
\usepackage[T2A]{fontenc}
\usepackage[english,russian]{babel}
\newenvironment{note}{\paragraph{NOTE:}}{}
\newenvironment{field}{\paragraph{field:}}{}

\usepackage{amsmath}
\usepackage{amsfonts}
\usepackage{amsthm}
\usepackage{amssymb}

\newtheorem{theorem}{Th}

\begin{document}

\section{Перестановки. Их свойства. Четная и нечетная перестановка.
Транспозиция.}

\begin{note}

  \begin{field}
    Перестановки
  \end{field}

  \begin{field}
    Пусть $\Omega$ --- множество, $|\Omega| = n$. Занумеровав эти элементы,
    считаем $\Omega = \{1, 2, \ldots, n\}$. Тогда перестановки --- биекции
    $\sigma : \Omega \rightarrow \Omega$.
    \begin{equation*}
      \sigma = \begin{pmatrix}
                 1 & 2 & \cdots & n \\
                 i_{1} & i_{2} & \cdots & i_{n}
               \end{pmatrix}
    \end{equation*}

    $S_{n}$ --- множество всех перестановок из $n$ элементов.

    Произведение перестановок --- $(\sigma \tau) = \sigma(\tau(i))$.
  \end{field}
\end{note}

\begin{note}
  \begin{field}
    Свойства перестановок
  \end{field}
  \begin{field}
    \begin{enumerate}
      \item
      $\sigma \tau \neq \tau \sigma$
      \item
      $(\sigma \tau) \pi = \sigma (\tau \pi)$ (т.к. это отображения)
      \item
      существует $e$ и $\sigma^{-1}$
    \end{enumerate}
  \end{field}
\end{note}

\begin{note}
  \begin{field}
    Четная и нечетная перестановка
  \end{field}
  \begin{field}
    $A_{n}$ --- множество четных перестановок.

    $(-1){}^{m}$ --- четность перестановки, где $m$ это количество транспозиций в
    разложении.

    Четность перестановки совпадает с четностью количества циклов четной длины.
  \end{field}
\end{note}

\begin{note}
  \begin{field}
    Транспозиция
  \end{field}
  \begin{field}
    Транспозиция --- цикл длины 2.
  \end{field}
\end{note}

\section{Разложение перестановки в произведение циклов. Представление
перестановки в виде произведения транспозиций. Представление четной перестановки
в виде произведения циклов длины 3.}

\begin{note}
  \begin{field}
    Разложение перестановки в произведение циклов
  \end{field}
  \begin{field}
    $\sigma \in S_{n}$. $i_{1} \in \Omega$. $i_{2} = \sigma(i_{1})$, $i_{3} = \sigma(i_{2})$, $\ldots$. Т.к. $|\Omega| = n$ и $\sigma$ --- биекция, $\exists i_{k}: i_{1} = \sigma(i_{k})$.

    \begin{equation*}
      (i_{1}i_{2}\ldots{} i_{k})
    \end{equation*}

    Далее из биекции находим остальные непересекающиеся циклы.
  \end{field}
\end{note}

\begin{note}
  \begin{field}
    Представление перестановки в виде произведения транспозиций.
  \end{field}

  \begin{field}
    Достаточно для одного цикла.

    \begin{equation*}
      (i_{1} i_{2} \cdots i_{k}) =
      (i_{1} i_{2})(i_{2} i_{3}) \cdots
      (i_{k-1} i_{k})
    \end{equation*}
  \end{field}
\end{note}

\begin{note}
  \begin{field}
    Представление четной перестановки в виде произведения циклов длины 3.
  \end{field}
  \begin{field}
    $\sigma \in A_{n} \Leftrightarrow \sigma разбивается на произведение циклов длины 3$.
  \end{field}
\end{note}

\section{Группы. Их свойства. Примеры. Подгруппы.}

\begin{note}
  \begin{field}
    Группы
  \end{field}
  \begin{field}
    $(G, \circ)$
    \begin{enumerate}
      \item
      ассоциативность
      \item
      $\exists e$
      \item
      $\forall a \in G \quad{} \exists a^{-1} \in G$
    \end{enumerate}

    Абелева $\Leftrightarrow \circ$ коммутативно

    $|G|$ --- порядок

    $a \in G, ord(a) = k$, если $a^{k} = e$ и $k$ --- минимально.

    $G$ --- $p$-группа $\Leftrightarrow$ $p$ --- простое и $|G| = p^{n}$.
  \end{field}
\end{note}

\begin{note}
  \begin{field}
    Свойства групп
  \end{field}
  \begin{field}
    \begin{enumerate}
      \item
      $e$ --- едиственна
      \item
      $a^{-1}$ --- единственнен
    \end{enumerate}
  \end{field}
\end{note}

\begin{note}
  \begin{field}
    Примеры групп
  \end{field}
  \begin{field}
    \begin{enumerate}
      \item
      $(\mathbb{Z}, +)$
      \item
      $(GL_{n}, \cdot)$ --- невырожденные матрицы
      \item
      $(S_{n}, \cdot)$
      \item
      $(A_{n}, \cdot)$
    \end{enumerate}
  \end{field}
\end{note}

\begin{note}
  \begin{field}
    Подгруппы
  \end{field}

  \begin{field}
    $(G, \circ)$. $H \subset G$ --- подгруппа $\Leftrightarrow$ $e \in H$, $H$ --- группа относительно $\circ$.

    $aH = \{ah : \forall h \in H\}$ --- левый смежный класс. Любая группа распадается на непересекающиеся смежные классы.
  \end{field}

  (Теорема Лагранжа): $|G| = n$, $H \subset G$ --- подгруппа, $|H| = k$. Тогда
  $n = k(G : H)$, где $(G : H)$ --- число смежных классов по $H$.

  Циклическая --- порождается одним элементом.

  $\exists g : a = g^{-1}bg$ --- $a$, $b$ сопряженные

  $G$ распадается на непересекающиеся классы сопряженности.
\end{note}

\section{Действия групп на множествах. Орбиты и стабилизаторы. Их свойства.
Формула Бернсайда.}

\begin{note}
  \begin{field}
    Дествия групп на множествах
  \end{field}
  \begin{field}
    $G$ --- группа, $M$ --- множество.

    Отбражение $G \times M \rightarrow M$:
    \begin{enumerate}
      \item
      $em = m \quad{} \forall m \in M$
      \item
      $g_{1}(g_{2}m) = (g_{1}g_{2})m$
    \end{enumerate}
  \end{field}
\end{note}

\begin{note}
  \begin{field}
    Орбиты и стабилизаторы
  \end{field}
  \begin{field}
    $St(m) = \{g \in G : gm = m\} \subset G$, подгруппа $G$

    $Orb(m) = \{gm : g \in G\} \subset M$.
  \end{field}
\end{note}

\begin{note}
  \begin{field}
    Свойства орбит и стабилизаторов
  \end{field}
  \begin{field}
    Имеется биекция $Orb(m) \mapsto \{g St(m) : g \in G\}$

    $|G| = |Orb(m)| \cdot |St(m)|$

    $m_{1}, m_{2} \in Orb(m) \Rightarrow St(m_{1}), St(m_{2})$ --- сопряжены.
  \end{field}
\end{note}

\begin{note}
  \begin{field}
    Формула Бернсайда
  \end{field}
  \begin{field}
    $|G| = n, |M| = m$.
    $M^{g} = \{m \in M : gm = m\}$.
    Тогда $N = \frac{1}{|G|} \sum_{g \ in G} |M^{g}|$, где $N$ --- число орбит.
  \end{field}
\end{note}

\section{Нормальные подгруппы. Гомоморфизмы групп. Ядро и факторгруппа. Первая
теорема о гомоморфизме.}

\begin{note}
  \begin{field}
    Нормальные подгруппы
  \end{field}
  \begin{field}
    $H \vartriangleleft G \Leftrightarrow g^{-1}Hg \subset H \quad{} \forall g \in G$

    (Или $gH = Hg \quad{} \forall g \in G$)
  \end{field}
\end{note}

\begin{note}
  \begin{field}
    Гомоморфизмы групп, ядро и факторгруппа
  \end{field}
  \begin{field}
    $f : G \rightarrow G'$:
    $f(ab) = f(a)f(b) \quad{} \forall a, b \in G$

    $\ker(f) = \{e\} \Leftrightarrow$ инъективный гомоморфизм

    $\forall g' \in G': \quad{} \exists g \in G, f(g) = g' \Leftrightarrow$ сюръективный гомоморфизм

    Сюръективный и инъективный гомоморфизм это изоморфизм ($G \cong G'$)

    $\ker(f) = \{ a \in G | f(a) = e \}$

    Если $f$ --- гомоморфизм:
    \begin{enumerate}
      \item
            $f(e) = e'$
      \item
            $f(a){}^{-1} = f(a^{-1})$
      \item
            $\ker(f) \vartriangleleft G$
    \end{enumerate}

    $G/H$ --- факторгруппа $\Leftrightarrow$ $G/H = \{gH \ | \ \forall g \in G, H \vartriangleleft G\}$.
    $aH \circ bH \mapsto abH$

  \end{field}
\end{note}

\begin{note}
  \begin{field}
    1-ая теорема о гомоморфизме групп
  \end{field}
  \begin{field}
    \begin{theorem}[1-ая теорема о гомоморфизме]
      $f$ --- сюръективный гомоморфизм $\Rightarrow \exists G/\ker(f) \cong G'$

      $H \vartriangleleft G \Rightarrow \quad{} \exists \varphi : G \rightarrow G/H : \varphi$ --- сюрьъекция, $\ker(\varphi) = H$
    \end{theorem}
  \end{field}
\end{note}

\section{Вторая и третья теоремы о гомоморфизме. Теорема Кели.}

\begin{note}
  \begin{field}
    2-ая теорема о гомоморфизме групп
  \end{field}
  \begin{field}
    \begin{theorem}[2-ая теорема о гомоморфизме групп]
      $H \subset G, K \vartriangleleft G \Rightarrow$
      \begin{enumerate}
        \item
              $HK = KH$
        \item
              $HK \subset G$
        \item
              $HK/K \cong H/(H \cap K)$
      \end{enumerate}
    \end{theorem}
  \end{field}
\end{note}

\begin{note}
  \begin{field}
    3-ая теорема о гомоморфизме групп
  \end{field}
  \begin{field}
    $H \vartriangleleft G, K \vartriangleleft G, K \subset H \Rightarrow G/H \cong (G/K)/(H/K)$
  \end{field}
\end{note}

\begin{note}
  \begin{field}
    Теорема Кели
  \end{field}
  \begin{field}
    \begin{theorem}[Теорема Кели]
      $|G| = n \Rightarrow \quad{} \exists f : G \rightarrow S_{n}$, $n$ --- инъективный гомоморфизм (вложение).
    \end{theorem}
  \end{field}
\end{note}

\section{Конечные абелевы группы. Их классификация.}

\begin{note}
  \begin{field}
    Конечные абелевы группы
  \end{field}
  \begin{field}
    В абелевой группе любая подгруппа нормальная.

    \begin{theorem}[Лемма]
      $G$ --- абелева группа, $\forall a \in G: ord(a) = p^{k}$, для какого-то
      $k \Rightarrow |G| = p^{n}$
    \end{theorem}

    \begin{theorem}
      $G$ --- абелева группа,
      $G = m = p_{1}^{s_{1}} \cdot p_{2}^{s_{2}} \ldots p_{n}^{s_{n}}$
      $\Rightarrow G \cong A_{p_{1}} \oplus \cdots \oplus A_{p_{n}}$, где
      $A_{p_{i}}$ --- абелева группа и $|A_{P_{i}} = p_{i}^{s_{i}}$
    \end{theorem}

    \begin{theorem}
      $G$ --- абелева $p$-группа
      $\Rightarrow G \cong \mathbb{Z}_{p_{s_{1}}} \oplus \cdots \oplus \mathbb{Z}_{p_{s_{n}}}$
    \end{theorem}

    \begin{theorem}
      $G$ --- абелева группа $\Rightarrow$ $G$ разлагается в прямую сумму
      циклических $p$-групп единственным образом.
    \end{theorem}
  \end{field}
\end{note}

\section{Свободные абелевы группы. Их базис. Классификация конечно порождённых
абелевых групп.}

\begin{note}
  \begin{field}
    Свободные абелевы группы
  \end{field}
  \begin{field}
    Без кручений $\Leftrightarrow \quad{} \nexists a \in A: na = 0$

    Конечно порожденная абелева группа $A$ без кручений называется свободной
    абелевой группой. Количество элементов в базисе --- ранг этой группы.
    Обозн. $F_{n}^{ab} = \mathbb{Z} + \cdots + \mathbb{Z}$ $n$ раз.
  \end{field}
\end{note}

\begin{note}
  \begin{field}
    Базис свободной абелевой группы
  \end{field}
  \begin{field}
    Система $a_{1}, a_{2}, \ldots, a_{k} \in A$ --- независимая
    $\Leftrightarrow (n_{1}a_{1} + n_{2}a_{2} + \cdots + n_{k}a_{k} = 0 , n_{i} \in \mathbb{Z} \Rightarrow n_{1} = n_{2} = \cdots = n_{k} = 0)$

    $A$ --- абелева без кручений.

    Система $a_{k}$ --- базис $A$ $\Leftrightarrow$ $a_{k}$ --- независимы и порождают $A$.

    Любая конечно порожденная абелева группа без кручений обладает базисом, притом все ее базисы равномощны.
  \end{field}
\end{note}

\begin{note}
  \begin{field}
    Классификация конечно порожденных абелевых групп
  \end{field}
  \begin{field}
    $A$ --- конечно порожденная абелева группа
    $Rightarrow A \cong F_{n}^{ab} \oplus B$, где $B$ --- конечная абелева группа.
  \end{field}
\end{note}

\section{Свободные группы. Задание группы образующими и соотношениями. Примеры.}

\begin{note}
  \begin{field}
    Свободные группы
  \end{field}
  \begin{field}
    $X = \{x_{i}\}$ --- множество символов (алфавит)

    $F(X)$ --- множество классов эквивалентных слов, где слово --- пустая или
    конечная последовательность символов из $X \cup X^{-1}$, а два слова
    эквивалентны, если после редукции (вычеркивание $x_{i}x_{i}^{-1}$ и
    $x_{i}^{-1}x_{i}$), получаем одинаковые слова. Класс эквивалентных слов обозн. $[u]$.

    На $F(X)$ введем операцию $[u][v] = [uv]$. Тогда $F(X)$ --- свободная
    группа. Если $X$ --- конечно, то конечно порожденная свободная группа.
    $|X|$ --- степень свободы.
  \end{field}
\end{note}

\begin{note}
  \begin{field}
    Задание группы образующими и соотношениями. Примеры
  \end{field}
  \begin{field}
    $G$ --- группа, порожденная $g_{1}, g_{2}, \ldots, g_{n}$.
    $X$ --- алфавит: $x_{1}, x_{2}, \ldots, x_{n}$.
    $\Rightarrow$
    $f : X \rightarrow M$, $f(x_{i}) = g_{i}$
    единственным образом продолжается до гомоморфизма групп
    $\overline{f}: F(X) \rightarrow G$:

    Элементы $\ker \overline{f}$ --- соотношения группы $G$ в алфавите $X$. Если множество соотношений $H'$ такого, что минимальная нормальная подгруппа в $F(X)$ содержащая $H'$, совпадает с $H$, то $H'$ --- определяющее множество соотношений

    $H' = \{x_{i}x_{j}x_{i}^{-1}x_{j}^{-1} | 1 \leq i < j \leq n\}$
    определяют свободную абелеву группу.
  \end{field}
\end{note}

\section{Кольца. Определение и основные свойства. Примеры.}

\begin{note}
  \begin{field}
    Кольца
  \end{field}
  \begin{field}
    $(A, +, \cdot)$ --- кольцо, если:
    \begin{enumerate}
      \item
      $(A, +)$ --- абелева группа
      \item
      $a(bc) = (ab)c$ и $\exists e$
      \item
      $(x + y)z = xz + yz$ и $z(x + y) = zx + zy$
    \end{enumerate}

    $ab = ba \Rightarrow$ коммутативное

    $\exists a^{-1} \Rightarrow$ тело

    тело, где $ab = ba$ это поле
  \end{field}
\end{note}

\begin{note}
  \begin{field}
    Свойства кольца
  \end{field}
  \begin{enumerate}
    \item
    $a \cdot 0 = 0 \cdot a = 0$
    \item
    $(-a)b = a(-b) = -(ab)$
    \item
    $(-a)(-b) = ab$
  \end{enumerate}
\end{note}

\begin{note}
  \begin{field}
    Примеры колец
  \end{field}
  \begin{field}
    $\mathbb{Z}$ --- коммутативное кольцо

    $\mathbb{Q}, \mathbb{R}, \mathbb{C}$ --- поля

    $M_{n \times m}$ некоммутативное кольцо

    $Z_{m} = \{0, 1, \ldots, m-1\}$ --- множество остатков при делении на $m$,
    коммутативное кольцо.
  \end{field}
\end{note}

\section{Идеалы. Факторкольца. Гомоморфизмы колец. Теорема о гомоморфизме для
колец. Прямое произведение колец. Группа единиц.}

\begin{note}
  \begin{field}
    Идеалы
  \end{field}
  \begin{field}
    $\mathbf{a} \subset A$ --- левый идеал кольца $A$
    $\Leftrightarrow$
    $\mathbf{a}$ --- подгруппа относительно сложения и
    $ax \in \mathbf{a} \forall a \in A, x \in \mathbf{a}$
    (т.е. $A\mathbf{a} \subset \mathbf{a}$)

    (Двусторонний) \emph{идеал} --- левый и правый идеал.

    $\mathbf{a}, \mathbf{b}$ --- идеалы $\Rightarrow$ $\mathbf{a} \cap \mathbf{b}$ --- тоже идеал
  \end{field}
\end{note}

\begin{note}
  \begin{field}
    Факторкольца
  \end{field}
  \begin{field}
    $A/\mathbf{a}$, элементы --- смежные классы $x + \mathbf{a}$.
    $(x + \mathbf{a})+(y + \mathbf{a}) = (x + y)+\mathbf{a}$,
    $(x + \mathbf{a})(y + \mathbf{a}) = xy+\mathbf{a}$
  \end{field}
\end{note}

\begin{note}
  \begin{field}
    Гомоморфизмы колец
  \end{field}
  \begin{field}
    $f : A \rightarrow B$:
    \begin{itemize}
      \item
      $f(a + b) = f(a) + f(b)$
      \item
      $f(ab) = f(a)f(b)$
    \end{itemize}

    $\ker(f) = \{a | f(a) = 0\}$
  \end{field}
\end{note}

\begin{note}
  \begin{field}
    Теорема о гомоморфизме для колец
  \end{field}
  \begin{field}
    \begin{theorem}[Теорема о гомоморфизме для колец]
      $f : A \rightarrow B$ --- сюръективный гомоморфизм колец
      $\Rightarrow \exists! A/\ker(f) \cong B$.

      $\mathbf{a} \subset A$ --- идеал кольца $A$
      $\Rightarrow \quad{} \exists! \varphi : a \rightarrow A/\mathbf{a} : \varphi$ --- сюръекция, $\ker(\varphi) = \mathbf{a}$
    \end{theorem}
  \end{field}
\end{note}

\begin{note}
  \begin{field}
    Прямое произведение колец
  \end{field}
  \begin{field}
    $A \times B = \{(a, b) | a \in A, b \in B\}$:
    \begin{itemize}
      \item
      $(a_{1}, b_{1}) + (a_{2}, b_{2}) = (a_{1} + a_{2}, b_{1} + b_{2})$
      \item
      $(a_{1}, b_{1})(a_{2}, b_{2}) = (a_{1}a_{2}, b_{1}b_{2})$
      \item
      ноль --- $(0, 0)$, $e$ --- $(1, 1)$
    \end{itemize}
  \end{field}
\end{note}

\begin{note}
  \begin{field}
    Группа единиц
  \end{field}
  \begin{field}
    $U \subset A$ --- множество обратимых элементов, называется группой единиц,
    а ее элементы --- единицами кольца $A$.
  \end{field}
\end{note}

\section{Коммутативные кольца. Максимальные и простые идеалы. Их свойства.
Критерий того, что факторкольцо является полем.}

\begin{note}
  \begin{field}
    Максимальный и простой идеал
  \end{field}
  \begin{field}
    $A$ --- коммутативное кольцо

    Идеал $\mathbf{p} \subset A$ --- простой
    $\Leftrightarrow$ ($xy \in \mathbf{p} \Rightarrow x \in \mathbf{p}$ либо $y \in \mathbf{p}$)

    $\mathbf{p}$ --- максимальный $\Leftrightarrow \nexists \mathbf{a} \neq A : \mathbf{m} \mathbf{a}$ и $\mathbf{m} \neq \mathbf{a}$
  \end{field}
\end{note}

\begin{note}
  \begin{field}
    Свойства максимального и простого идеала
  \end{field}
  \begin{field}
    \begin{enumerate}
      \item
            $\mathbf{p} \subset A$ --- простой
            $\Leftrightarrow A/\mathbf{p}$ --- целостно
      \item
            всякий максимальный идеал --- простой
      \item
            $f : A \rightarrow B$ --- гомоморфизм,
            $\mathbf{p'}$ --- простой идеал кольца $B$
            $\Rightarrow \mathbf{p} = f^{-1}(\mathbf{p'})$ ---
            простой идеал кольца $A$
    \end{enumerate}
  \end{field}
\end{note}

\begin{note}
  \begin{field}
    Критерий того, что факторкольцо является полем
  \end{field}
  \begin{field}
    $A/\mathbf{m}$ --- поле
    $\Leftrightarrow \mathbf{m}$ --- максимальный идеал
    кольца $A$
  \end{field}
\end{note}

\section{Китайская теорема об остатках. Ее следствия.}

\begin{note}
  \begin{field}
    Китайская теорема об остатках
  \end{field}
  \begin{field}
    $A$ --- коммутативное кольцо,
    $\mathbf{a_{1}}, \mathbf{a_{2}}, \ldots, \mathbf{a_{n}}$ --- идеалы $A$, $\mathbf{a_{i}} + \mathbf{a_{j}} = A \forall i \neq j$
    $\Rightarrow \forall$ семейства $x_{1}, x_{2}, \ldots, x_{n} \in A \exists x \in A: x \equiv x_{i} (\mod \mathbf{a_{i}}) \forall i$

    $x \equiv y (\mod \mathbf{a}) \Leftrightarrow x - y \in \mathbf{a}$

  \end{field}
\end{note}

\begin{note}
  \begin{field}
    Следствия из китайской теоремы об остатках
  \end{field}
  \begin{field}
    \begin{theorem}[Следствие]
      $A$ --- коммутативное кольцо,
    $\mathbf{a_{1}}, \mathbf{a_{2}}, \ldots, \mathbf{a_{n}}$ --- идеалы $A$,
    $\mathbf{a_{i}} + \mathbf{a_{j}} = A \forall i \neq j$,
    $f : A \rightarrow (A/ \mathbf{a_{1}}) \ times (A/ \mathbf{a_{2}}) \times \cdots \times (A/ \mathbf{a_{n}})$
    --- отображение, индуцированное каноническими отображениями $A$ в $A/ \mathbf{a_{i}}$ для каждого множителя
    $\Rightarrow \ker f = \mathbf{a_{1}} \cap \mathbf{a_{2}} \cap \cdots \cap \mathbf{a_{n}}$ и
    $A/(\mathbf{a_{1}} \cap \mathbf{a_{2}} \cap \cdots \cap \mathbf{a_{n}}) \cong (A/\mathbf{a_{1}}) \times (A/\mathbf{a_{2}}) \times \cdots \times (A/\mathbf{a_{n}})$
    \end{theorem}

    \begin{theorem}[Для целых чисел]
      $m_{1}, m_{2}, \ldots, m_{n}$ --- попарно взаимно простые целые числа
      $\Rightarrow \forall x_{1}, x_{2}, \ldots, x_{n} \exists x \in \mathbb{Z}: x \equiv x_{i} (\mod m_{i}) \forall i$
    \end{theorem}
  \end{field}
\end{note}

\section{Главные идеалы. Кольцо главных идеалов. Примеры. Целостные и
факториальные кольца.}

\begin{note}
  \begin{field}
    Главный идеал
  \end{field}
  \begin{field}
    $\mathbf{a}$ --- главный
    $\Leftrightarrow \exists a \in A: \mathbf{a} = aA$

    Обозн. $\mathcal{a} = (a)$

    $(m)$ --- идеал $\mathbb{Z}$ $\Rightarrow \mathbb{Z}/(m) = /mathbb{Z}_{m}$ и
    существует естественное отображение
    $f : \mathbb{Z} \rightarrow \mathbb{Z}_{m}$
  \end{field}
\end{note}

\begin{note}
  \begin{field}
    Кольцо главных идеалов
  \end{field}
  \begin{field}
    Кольцо главных идеалов $\Leftrightarrow$
    все идеалы главные

    $\mathbb{Z}$ --- кольцо главных идеалов
  \end{field}
\end{note}

\begin{note}
  \begin{field}
    Целостное кольцо
  \end{field}
  \begin{field}
    целостное кольцо $\Leftrightarrow$ $\nexists a, b: ab=0$

    (или кольцо без делителей нуля)
  \end{field}
\end{note}

\begin{note}
  \begin{field}
    Факториальное кольцо
  \end{field}
  \begin{field}
    $A$ --- факториальное $\Leftrightarrow$ целостное и всякий элемент имеет
    однозначное разложение на неприводимые.

    $a \neq 0, a \in A (целостное), (a)$ --- простой главный идеал
    $\Leftrightarrow$ $a$ --- неприводим.

    $\forall a \in A, a \neq 0$ обладает однозначным разложением на
    неприводимые, если: $\exists u$ --- единица и неприводимые элементы
    $p_{1}, p_{2}, \ldots, p_{k}$ такие, что $a = up_{1}p_{2}\cdots p_{k}$.
    Причем для двух таких разложений
    $a = u p_{1}p_{2}\cdots p_{k} = u' q_{1} q_{2} \cdots q_{m}$,
    $m = k$ и, с точностью до перестановки,
    $q_{i} = u_{i}p_{i}$, где $u_{i}$ --- единицы в $A$.
  \end{field}
\end{note}

\section{НОД. Теорема о том, что любое кольцо главных идеалов факториально.
Примеры.}

\begin{note}
  \begin{field}
    НОД
  \end{field}
  \begin{field}
    $a$ делит $b$
    $\Leftrightarrow$
    $\exists c \in A: b = ac$.

    $d$ --- НОД $a$ и $b$
    $\Leftrightarrow$
    любой $c$, делящий $a$ и $b$, делит также $d$.
  \end{field}
\end{note}

\begin{note}
  \begin{field}
    Теорема о том, что любое кольцо главных идеалов факториально.
  \end{field}
  \begin{field}
    Всякое целостное кольцо главных идеалов факториально.

    Пример: $\mathbb{Z}$, группа единиц состоит из $1$ и $-1$.
    Неприводимые --- простые числа.

    $\mathbb{R}[x], \mathbb{Q}[x]$
  \end{field}
\end{note}

\section{Локализация. Ее свойства. Примеры.}

\begin{note}
  \begin{field}
    Локализация
  \end{field}
  \begin{field}
    $A$ --- коммутативное кольцо,
    $S \subset A$ --- мультипликативное подмножество,
    если $1 \in S$ и $x, y \in S \Rightarrow xy \in S$.

    $(a, s) \equiv (a', s') \Leftrightarrow \exists s'' \in S: s''(as' - sa') = 0$ --- отношение эквивалентности.

    $S^{-1}A$ --- множество классов эквивалентности (кольцо). Эл-ты: $\frac{a}{s}$, $\frac{s}{s}$ --- единица.

    $(\frac{a}{s})(\frac{a'}{s'}) = \frac{aa'}{ss'}$

    $\frac{a}{s} + \frac{a'}{s'} = \frac{as' + a's}{ss'}$
  \end{field}
\end{note}

\begin{note}
  \begin{field}
    Свойства локализации
  \end{field}
  \begin{field}
    $\varphi_{S}A \rightarrow S^{-1}A$, $\varphi_{S}(a) = \frac{a}{1}$

    $a$ --- целостное кольцо, $S$ --- мультипликативное множество, не содержащее нуля.
    $\Rightarrow$ $\varphi_{S}$ --- инъективный гомоморфизм

    $S \subset S^{-1}A$ обратимо
  \end{field}
\end{note}

\begin{note}
  \begin{field}
    Примеры локализации
  \end{field}
  \begin{field}
    $A$ --- целостное кольцо, $S$ состоит из обратимых элементов
    $\Rightarrow$ $S^{-1}A = A$

    $A$ --- целостное кольцо, $S$ --- множество его ненулевых элементов
    $\Rightarrow$ $S$ --- мультипликативное множество,
    $S^{-1}A$ --- поле (назыв. \emph{полем частных} кольца $A$).

    Например, $\mathbb{Q}$ --- поле частных кольца $\mathbb{Z}$
  \end{field}
\end{note}

\section{Многочлены. Определения свойства. Трансцендентные и алгебраические
  элементы.}

% 36

\begin{note}
  \begin{field}
    Многочлены
  \end{field}

  \begin{field}
    $A$ --- коммутативное кольцо. Построим кольцо $B$, эл-ты:
    $f = (a_{0}, a_{1}, \ldots, a_{n}, \ldots), a_{i} \in A$,
    где конечное число $a_{i} \neq 0$

    $f + g = (a_{0}, \ldots, a_{n}, \ldots) + (b_{0}, \ldots, b_{n}, \ldots) = (a_{0} + b_{0}, \ldots, a_{n} + b_{n}, \ldots)$

    $f \cdot g = (a_{0}, \ldots, a_{n}, \ldots) + (b_{0}, \ldots, b_{n}, \ldots) = (h_{0}, \ldots, h_{n}, \ldots)$,
    где $h_{m} = \sum_{i + j = m} a_{i}b_{j}$

    $\varphi A \rightarrow B$ --- инъективный гомоморфизм,
    $\varphi(a) = (a, 0, 0, \ldots)$. А значит $A \subset B$

    $x = (0, 1, 0, 0, 0, \ldots)$

    $x^{2} = (0, 0, 1, 0, 0, \ldots)$

    $x^{3} = (0, 0, 0, 1, 0, \ldots)$

    $a \cdot x^{n} = a \cdot (0, 0, \ldots, 0, 1, 0, \ldots) = (0, 0, \ldots, 0, a, 0, \ldots)$

    $f(x) = a_{0} + a_{1}x + a_{2} x^{2} + \cdots + a_{n}x^{n}$

    Кольцо $B$ --- \emph{кольцо многочленов от одной переменной}.
    Обозн. $A[x]$. $a_{0}, \ldots$ --- \emph{коэффициенты} $f(x)$.
    $\deg(f)$ --- максимальное $n$, для которого $a_{n} \neq 0$
  \end{field}
\end{note}

\begin{note}
  \begin{field}
    Свойства многочленов
  \end{field}
  \begin{field}
    $A$ --- целостное кольцо
    $\Rightarrow$
    $\deg (f + g) \leq \max(\deg(f), \deg(g)),
    \deg(fg) = \deg(f) + \deg(g)$


    $A$ --- целостное кольцо
    $\Rightarrow$
    $A[x]$ --- тоже целостное

    $A$ --- подкольцо коммутативного кольца $K$,
    $\alpha \in K$
    $\Rightarrow$
    $\exists! $ гомоморфизм $\varphi_{\alpha} : A[x] \rightarrow K$,
    $\varphi_{\alpha}(a) = a \forall a \in A$,
    $\varphi_{\alpha}(x) = \alpha$
  \end{field}
\end{note}

\begin{note}
  \begin{field}
    Алгебраические и трансцендентные элементы
  \end{field}
  \begin{field}
    $\alpha$ --- алгебраический над $A$
    $\Leftrightarrow$
    $\exists f \in A[x]: \varphi_{\alpha}(f) = 0$

    Если $\varphi_{\alpha}$ --- инъективно, то трансцендентным.

    $\varphi_{\alpha} (f) = 0$ значит $\alpha$ корень $f$. Обозн. $f(\alpha) = 0$.
  \end{field}
\end{note}

\section{Алгоритм Евклида. Евклидовы кольца.}

% \begin{note}
%   \begin{field}
%     Алгоритм Евклида
%   \end{field}
%   \begin{field}
%     TODO
%   \end{field}
% \end{note}

\begin{note}
  \begin{field}
    Евклидовы кольца
  \end{field}
  \begin{field}
    $A$ --- евклидово кольцо
    $\Leftrightarrow$
    $A$ --- целостное кольцо,
    $\delta : A \ {0} \rightarrow \mathbb{N} \cup {0}$
  \end{field}
\end{note}

\section{Теорема о том, что любое евклидово кольцо является кольцом главных
  идеалов. Лемма Гаусса.}

\begin{note}
  \begin{field}
    Теорема о том, что любое евклидово кольцо является кольцом главных
    идеалов.
  \end{field}
  \begin{field}
    Всякое евклидово кольцо является кольцом главных идеалов

    Всякое евклидово кольцо является факториальным
  \end{field}
\end{note}

\begin{note}
  \begin{field}
    Лемма Гаусса
  \end{field}
  \begin{field}
    $f(x), g(x)$ --- многочлены с целыми коэффициентами.
    $a = \gcd(a_{i}), b = \gcd(b_{i})$,
    $c = \gcd(c_{i})$, где $c_{i}$ --- коэффициенты $f(x)g(x)$.
    $\Rightarrow c = ba$
  \end{field}
\end{note}

\section{Неприводимые многочлены. Расширение полей. Алгебраически замкнутые
поля.}

\begin{note}
  \begin{field}
    Неприводимые многочлены
  \end{field}
  \begin{field}
    $f \in K[x], \deg(f) \neq 0$ --- \emph{неприводимый} над полем $K$
    $\Leftrightarrow$
    $\nexists g \in K[x], 1 \leq \deg(g) < \deg(f), f$ делится на $g$
  \end{field}
\end{note}

\begin{note}
  \begin{field}
    Расширение полей
  \end{field}
  \begin{field}
    Поле $K[x]/(f)$ называется расширением поля $K$,
    где $f$ --- неприводимый многочлен над $K$

    TODO дописать?
  \end{field}
\end{note}

\begin{note}
  \begin{field}
    Алгебраически замкнутые поля
  \end{field}
  \begin{field}
    Поле $K$ --- \emph{алгебраически замкнутое}
    $\Leftrightarrow$
    $\forall f(x) \in K[x]$ имеет корень
  \end{field}
\end{note}

\section{Основная теорема алгебры (алгебраическая замкнутость поля комплексных
чисел).}

\begin{note}
  \begin{field}
    Основная теорема алгебры
  \end{field}
  \begin{field}
    Поле $\mathbb{C}$ алгебраически замкнуто

    TODO леммы
  \end{field}
\end{note}

% 42
\section{Модули. Определение и примеры. Основные свойства. Векторное
пространство, как модуль.}

\begin{note}
  \begin{field}
    Модули
  \end{field}
  \begin{field}
    $A$ --- кольцо.
    \emph{Левый модуль} (модуль) над $A$ ---
    абелева группа $M$ c действием $A$ на $M$, если:
    \begin{enumerate}
      \item
      $(a + b)x = ax + bx$
      \item
      $a(x + y) = ax + ay$
      \item
      $(ab)x = a(bx)$
      \item
      $1 \cdot x = x$
    \end{enumerate}
    $a, b \in A; x, y \in M$

    Подгруппа $N \subset M$ --- подмодуль
    $\Leftrightarrow$
    $AN \subset N$
  \end{field}
\end{note}

\begin{note}
  \begin{field}
    Примеры модулей
  \end{field}
  \begin{field}
    \begin{itemize}
      \item
      любой левый идеал
      \item
      любая коммутативная группа есть $\mathbb{Z}$-модуль
      \item
      $A[x]$ есть $A$-модуль
    \end{itemize}
  \end{field}
\end{note}

% \begin{note}
%   \begin{field}
%     Основные свойства модулей
%   \end{field}
%   \begin{field}
%     TODO
%   \end{field}
% \end{note}

\begin{note}
  \begin{field}
    Векторное пространство, как модуль.
  \end{field}
  \begin{field}
    Модуль $M$ над полем называется \emph{векторным пространством}.

    $M$ конечно порожден $\Leftrightarrow$
    $M$ --- конечномерное векторное пространство.
  \end{field}
\end{note}

\section{Теоремы о гомоморфизме для модулей. Аннулятор.}

\begin{note}
  \begin{field}
    1-я теорема о гомоморфизме для модулей
  \end{field}
  \begin{field}
    $f : M \rightarrow M'$ --- сюръективный гомоморфизм модулей
    $\Rightarrow \exists$ естественный изоморфизм
    $M/ \ker(f) \cong M'$
  \end{field}
\end{note}

\begin{note}
  \begin{field}
    2-я теорема о гомоморфизме для модулей
  \end{field}
  \begin{field}
    $N, N'$ --- подмодули $M$
    $\Rightarrow$
    $(N + N')/N' \cong N/(N \cap N')$
  \end{field}
\end{note}

\begin{note}
  \begin{field}
    3-я теорема о гомоморфизме для модулей
  \end{field}
  \begin{field}
    $N, N'$ --- подмодули $M$,
    $N' \subset N$
    $\Rightarrow$
    $M/N \cong (M/N')/(N/N')$
  \end{field}
\end{note}

\begin{note}
  \begin{field}
    Аннулятор
  \end{field}
  \begin{field}
    $\mathsf{Ann}(M) = \{a | a \in A, ax = 0, \forall x \in M\}$

    Модуль $M$ --- точный $\Leftrightarrow \mathsf{Ann}(M) = 0$

    $\mathsf{Ann}(M)$ --- двусторонний идеал кольца $A$
  \end{field}
\end{note}

% 44
\section{Алгебры. Определения и примеры. Аналог теоремы Кели для алгебр.}

\begin{note}
  \begin{field}
    Алгебры
  \end{field}
  \begin{field}
    $A$ --- алгебра над полем $K$ (или $K$-алгебра)
    $\Leftrightarrow$ $A$ --- векторное пространство над $K$
    и на $A$ есть умножение:
    \begin{enumerate}
      \item
      $x(y + z) = xy + xz$
      \item
      $(x + y)z = xz + yz$
      \item
      $(ax)y = x(ay) = a(xy)$
    \end{enumerate}

    $\forall x, y \in A, a \in K$

    Ассоциативная алгебра,
    Алгебра с единицей
  \end{field}
\end{note}

\begin{note}
  \begin{field}
    Примеры алгебр
  \end{field}
  \begin{field}
    $\mathbb{C}$ --- алгебра над $\mathbb{R}$

    $K$ --- поле $\Rightarrow$
    $K[x]$ --- $K$-алгебра

    $M_{n}(K)$ --- кольцо матриц
  \end{field}
\end{note}

\begin{note}
  \begin{field}
    Аналог теоремы Кели для алгебр.
  \end{field}
  \begin{field}
    $A$ --- $n$-мерная алгебра над полем $K$
    $\Rightarrow$ $A$ изоморфна некоторой подалгебре в $M_{n}(K)$
  \end{field}
\end{note}

\section{Конечномерные алгебры. Минимальный многочлен элемента. Алгебры с
делением. Обратимость элемента, не являющегося делителем нуля.}

\begin{note}
  \begin{field}
    Конечномерные алгебры
  \end{field}
  \begin{field}
    Алгебра $A$ над полем $K$ --- конечномерная
    $\Leftrightarrow$
    конечномерно векторное пространство $A$ над $K$.

    Алгебра $A$ над полем $K$ --- конечно порожденная
    $\Leftrightarrow$
    существует конечное множество элементов пораждающих $A$
  \end{field}
\end{note}

\begin{note}
  \begin{field}
    Минимальный многочлен элемента
  \end{field}
  \begin{field}
    Многочлен $f(x) \in K[x]$ для которого $f(a) = 0$
    называется аннулирующим элемент $a$

    $A$ --- конечномерная алгебра над полем $K$, $a \in A$ $\Rightarrow$ $f(x)$
    аннулирующий $a$, степень которого минимальна и старший коэффициент равен 1,
    называется минимальным многочленом элемента $a$ над $K$
  \end{field}
\end{note}

\begin{note}
  \begin{field}
    Алгебры с делением
  \end{field}
  \begin{field}
    $K$-алгебра $A$, $1 \in A$, любой элемент обратим
    $\Rightarrow$ $A$ --- алгебра с делением

    \begin{theorem}
      $A$ --- конечномерная ассоциативная коммутативная алгебра с делением над полем $\mathbb{R}$ (т.е. $A$ --- поле)
      $\Rightarrow$
      либо $A = \mathbb{R}$, либо $A = \mathbb{C}$
    \end{theorem}
  \end{field}
\end{note}

\begin{note}
  \begin{field}
    Обратимость элемента, не являющегося делителем нуля
  \end{field}
  \begin{field}
    \begin{theorem}
      $A$ --- конечномерная алгебра над полем $K$.
      $\Rightarrow$
      $\forall a \in A, a$ либо обратим, либо является делителем нуля.
    \end{theorem}
  \end{field}
\end{note}

\section{Задание алгебры. Тело кватернионов. Теорема Фробениуса (б/д).}

\begin{note}
  \begin{field}
    Задание алгебры
  \end{field}
  \begin{field}
    $A$ --- конечномерная алгебра над полем $K$
    и $e_{1}, e_{2}, \ldots, e_{n}$ --- базис $A$ над $K$.
    $\Rightarrow$
    Соотношения $e_{i}e_{j} = \sum_{k=1}^{n} g_{ij}^{k} e_{k}$
    задают структуру $K$-алгебры на $A$.

    $a = a_{1}e_{1} + a_{2}e_{2} + \cdots + a_{n}e_{n}$,
    $b = b_{1}e_{1} + b_{2}e_{2} + \cdots + b_{n}e_{n}$
    $\Rightarrow$
    $ab = \left(\sum_{i=1}^{n} a_{i}e_{i}\right) \left(\sum_{j=1}^{n} b_{j}e_{j}\right) =$
    $\sum_{i=1}^{n}\sum_{j=1}^{n}a_{i}b_{j}e_{i}e_{j} =$
    $\sum_{i=1}^{n}\sum_{j=1}^{n}\sum_{k=1}^{n}a_{i}b_{j}g_{ij}^{k}e_{k}$
  \end{field}
\end{note}

% \begin{note}
%   \begin{field}
%     Тело кватерионов
%   \end{field}
%   \begin{field}
%     TODO
%   \end{field}
% \end{note}

\begin{note}
  \begin{field}
    Теорема Фробениуса
  \end{field}
  \begin{field}
    \begin{theorem}[Фробениус]
      $A$ --- конечномерная ассоциативная алгебра с делением над полем $\mathbb{R}$.
      $\Rightarrow$
      либо $A = \mathbb{R}$,
      либо $A = \mathbb{C}$,
      либо $A = \mathbb{H}$
    \end{theorem}
  \end{field}
\end{note}

\section{Алгебры с делением над полем комплексных чисел. Теорема Фробениуса
(коммутативный случай). Групповая алгебра. Дифференцирование алгебр.}

% \begin{note}
%   \begin{field}
%     Алгебры с делением над полем комплексных чисел
%   \end{field}
%   \begin{field}
%     TODO
%   \end{field}
% \end{note}

\begin{note}
  \begin{field}
    Теорема Фробениуса (коммутативный случай)
  \end{field}
  \begin{field}
    \begin{theorem}[Фробениус (коммутативный случай)]
      $A$ --- конечномерная ассоциативная коммутативная алгебра с делением над полем $\mathbb{R}$ (т.е. $A$ --- поле).
      $\Rightarrow$
      либо $A = \mathbb{R}$,
      либо $A = \mathbb{C}$.
    \end{theorem}
  \end{field}
\end{note}

% \begin{note}
%   \begin{field}
%     Групповая алгебра
%   \end{field}
%   \begin{field}
%     TODO
%   \end{field}
% \end{note}

\begin{note}
  \begin{field}
    Дифференцирование алгебр
  \end{field}
  \begin{field}
    $A$ --- алгебра над полем $K$.
    Дифференциерование алгебры это $d : A \rightarrow A$, если:
    \begin{enumerate}
      \item
      $d (ax) = adx$
      \item
      $d (x + y) = dx + dy$
      \item
      $d (xy) = (dx) y + x (dy)$
    \end{enumerate}

    $\forall x, y \in A, a \in K$
  \end{field}
\end{note}

% 50
\end{document}
