% -*- coding:utf-8 -*-
\documentclass[12pt]{article}
\usepackage[utf8]{inputenc}
\usepackage[T2A]{fontenc}
\usepackage[english,russian]{babel}
\newenvironment{note}{\paragraph{NOTE:}}{}
\newenvironment{field}{\paragraph{field:}}{}

\usepackage{amsmath}
\usepackage{amsfonts}
\usepackage{amsthm}
\usepackage{amssymb}

\newtheorem{theorem}{Th}

\begin{document}

\begin{note}
  \begin{field}
    Расширение поля
  \end{field}
  \begin{field}
    $E, k$ --- два поля,
    $k \subset E$
    $\Leftrightarrow$
    $E$ --- \emph{расширение} $k$
  \end{field}
\end{note}

\begin{note}
  \begin{field}
    Конечное (бесконечномерное) расширение
  \end{field}
  \begin{field}
    Расширение $E$ конечно (бесконечно) $\Leftrightarrow$ $E$ --- конечномерно
    (бесконечномерно) как линейное пространство над $k$

    Т.е. $E$ --- конечно над $k$
    $\Leftrightarrow$
    $\exists a_{1}, \ldots, a_{n} \in E : \forall x \in E$,
    $x = \alpha_{1}a_{1} + \cdots + \alpha_{n} a_{n}$,
    где $\alpha_{1}, \ldots \in k$.

    $[E : k]$ (степень $E$ над $k$) ---
    размерность $E$ как линейного пространства
  \end{field}
\end{note}

\begin{note}
  \begin{field}
    Теорема о конечных расширениях полей
  \end{field}
  \begin{field}
    \begin{theorem}
      Пусть $E$ --- конечное расширение поля $k$,
      $F$ --- конечное расширение поля $E$.
      $\Rightarrow$
      $F$ --- конечное расширение поля $k$
      и $[F : k] = [E : k][F : E]$.
    \end{theorem}
  \end{field}
\end{note}

\begin{note}
  \begin{field}
    Алгебраический элемент
  \end{field}
  \begin{field}
    $x \in E$ --- алгебраический
    $\Leftrightarrow$
    $x$ является корнем многочлена с коэффициентами из $k$
    ($\exists \alpha_{0}, \ldots, \alpha_{n} \in k :
    \alpha_{0} + \alpha_{1}x + \cdots \alpha_{n}x^{n} = 0$)

    Расширение $E$ поля $k$ --- алгебраическое
    $\Leftrightarrow$
    $\forall e \in E : e$ --- алгебраический.
  \end{field}
\end{note}

\begin{note}
  \begin{field}
    Теорема о алгебраических расширениях конечных полей
  \end{field}
  \begin{field}
    Любое конечное расширение является алгебраическим
  \end{field}
\end{note}

\begin{note}
  \begin{field}
    Теорема о наименьшем подполе
  \end{field}
  \begin{field}
    $E$ --- расширение поля $k$.
    $a_{1}, \ldots, a_{n} \in E$.
    $k(a_{1}, \ldots, a_{n})$ --- наименьшее подполе $E$,
    содержащее $a_{1}, \ldots, a_{n}$.

    \begin{theorem}
      $a \in E$ алгебраичен над $k$
      $\Rightarrow k(a)$ --- конечное расширение поля $k$
    \end{theorem}
  \end{field}
\end{note}

\begin{note}
  \begin{field}
    Теорема о алгебраических расширениях
  \end{field}
  \begin{field}
    \begin{theorem}
      $E$ --- алгебраическое расширение поля $k$
      и $F$ --- алгебраическое расширение поля $E$
      $\Rightarrow F$ --- алгебраическое расширение поля $k$.
    \end{theorem}
  \end{field}
\end{note}

\begin{note}
  \begin{field}
    Теорема о существовании расширения, в котором многчлен имеет корень
  \end{field}
  \begin{field}
    \begin{theorem}
      $\forall p(x) \in k[x] \exists$
      расширение поля $k$ в котором $p(x)$ имеет корень
    \end{theorem}
  \end{field}
\end{note}

\begin{note}
  \begin{field}
    Характеристика поля
  \end{field}
  \begin{field}
    $k$ --- поле

    Пусть $\exists p : p \cdot 1 = 0$
    и $p$ --- минимально,
    $\Rightarrow$
    $p = char(k)$.

    Если такого $p$ не существует, то $char(k) = 0$
  \end{field}
\end{note}

\begin{note}
  \begin{field}
    Свойства характеристики
  \end{field}
  \begin{field}
    Простое число или ноль

    $(a + b)^{p} = a^{p} + b^{p}$
  \end{field}
\end{note}

\begin{note}
  \begin{field}
    Морфизм Фробениуса
  \end{field}
  \begin{field}
    $(a + b)^{p} = a^{p} + b^{p}$,
    $(ab)^{p} = a^{p}b^{p}$
    $\Rightarrow$
    $f : k \mapsto k^{p}, f(x) = x^{p}$ --- гомоморфизм
  \end{field}
\end{note}

\begin{note}
  \begin{field}
    Совершенное поле
  \end{field}
  \begin{field}
    $char(k) = 0$ или $char(k) = p$, $k = k^{p}$
  \end{field}
\end{note}

\begin{note}
  \begin{field}
    Признак совершенного поля
  \end{field}
  \begin{field}
    $k$ --- конечное поле
    $\Rightarrow$
    $k$ --- совершенное поле
  \end{field}
\end{note}

\end{document}
